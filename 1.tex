
%%% Local Variables: 
%%% mode: latex
%%% TeX-master: "weka"
%%% End: 
\section{OLAP}
\subsection{OLAP相关概念}
Mondrian是一个基于Java语言的开源OLAP引擎,它通过MDX语句执行查询,从关系型数据库RDBMS中读取数据,以多维度的形式展示查询结果。
\par Mondrian通过Schema来定义一个多维数据库,它是一个逻辑概念上的模型,其中包含Cube(立方体)、Dimension(维度)、Hierarchy(层次)、Level(级别)、Measure(度量),这些被映射到数据库物理模型。Mondrian中Schema是以XML文件的形式定义的。
\begin{itemize}
\item Cube(立方体)是一系列Dimension和Measure的集合区域,它们共用一个事实表。
\item Dimension(维度)是一个Hierarchy的集合,维度一般有其相对应的维度表,它由Hierarchy(层次)组成,而Hierarchy(层次)又是由组成Level(级别)的。
\item Hierarchy(层次)是指定维度的层级关系的,如果没有指定,默认Hierarchy里面装的是来自立方体中的真实表。
\item Level(级别)是Hierarchy的组成部分,使用它可以构成一个结构树,Level的先后顺序决定了Level在结构树上的位置,最顶层的 Level 位于树的第一级,依次类推。
\item Measure(度量)是我们要进行度量计算的数值,支持的操作有sum、count、avg、distinct-count、max、min等。
\end{itemize}
\par 概括总结一下:在多维分析中,关注的内容通常被称为度量(Measure),而把限制条件称为维度(Dimension)。多维分析就是对同时满足多种限制条件的所有度量值做汇总统计。包含度量值的表被称为事实表(Fact Table),描述维度具体信息的表被称为维表(Dimension Table),同时有一点需要注意:并不是所有的维度都要有维表,对于取值简单的维度,可以直接使用事实表中的一列作为维度展示。
\par 下面是Mondrian中一个简单的Schema文件,其中包含一个名为“Sales”的Cube,立方体中有两个维度:“Gender”和“Time”,两个度量值:“Unit Sales”和“Store Sales”。
\begin{verbatim}
<Schema>
<Cube name="Sales">
<Table name="sales_fact_1997"/>
<Dimension name="Gender" foreignKey="customer_id">
<Hierarchy hasAll="true" allMemberName="All Genders" primaryKey="customer_id">
<Table name="customer"/>
<Level name="Gender" column="gender" uniqueMembers="true"/>
</Hierarchy>
</Dimension>
<Dimension name="Time" foreignKey="time_id">
<Hierarchy hasAll="false" primaryKey="time_id">
<Table name="time_by_day"/>
<Level name="Year" column="the_year" type="Numeric" uniqueMembers="true"/>
<Level name="Quarter" column="quarter" uniqueMembers="false"/>
<Level name="Month" column="month_of_year" type="Numeric" uniqueMembers="false"/>
</Hierarchy>
</Dimension>
<Measure name="Unit Sales" column="unit_sales" aggregator="sum" formatString="#,###"/>
<Measure name="Store Sales" column="store_sales" aggregator="sum" formatString="#,###.##"/>
<Measure name="Store Cost" column="store_cost" aggregator="sum" formatString="#,###.00"/>
<CalculatedMember name="Profit" dimension="Measures" formula="[Measures].[Store Sales] - [Measures].[Store Cost]">
<CalculatedMemberProperty name="FORMAT_STRING" value="$#,##0.00"/>
</CalculatedMember>
</Cube>
</Schema>
\end{verbatim}
\subsection{数据预处理}
\begin{itemize}
\item 数据清理用来清除数据中的噪声,纠正不一致。
\item 数据集成将数据由多个数据源合并成一个一致的数据存储,如数据仓库。
\item 数据归约可以通过聚集,删除冗余特征或者聚类方法来降低数据的规模,便于分析。
\item 数据变换可以用来把数据压缩到较小的空间,如0.0到1.0,可以提高涉及距离度量的挖掘算法的准确率和效率。
\end{itemize}
\par 不正确,不完整和不一致的数据是商场中数据的普遍特点,如许多数据库表元组在某些属性上根本没有值,有些属性没有被记录,对同一款商品,不同商场的命名不一样。
