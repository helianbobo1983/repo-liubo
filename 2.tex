\section{OLAP}
\subsection{OLAP相关概念(联机分析处理)}
Mondrian是一个基于Java语言的开源OLAP引擎,它通过MDX语句执行查询,从关系型数据库RDBMS中读取数据,以多维度的形式展示查询结果。
\par Mondrian通过Schema来定义一个多维数据库,它是一个逻辑概念上的模型,其中包含Cube(立方体)、Dimension(维度)、Hierarchy(层次)、Level(级别)、Measure(度量),这些被映射到数据库物理模型。Mondrian中Schema是以XML文件的形式定义的。
\begin{itemize}
\item Cube(立方体)是一系列Dimension和Measure的集合区域,它们共用一个事实表。
\item Dimension(维度)是一个Hierarchy的集合,维度一般有其相对应的维度表,它由Hierarchy(层次)组成,而Hierarchy(层次)又是由组成Level(级别)的。
\item Hierarchy(层次)是指定维度的层级关系的,如果没有指定,默认Hierarchy里面装的是来自立方体中的真实表。
\item Level(级别)是Hierarchy的组成部分,使用它可以构成一个结构树,Level的先后顺序决定了Level在结构树上的位置,最顶层的 Level 位于树的第一级,依次类推。
\item Measure(度量)是我们要进行度量计算的数值,支持的操作有sum、count、avg、distinct-count、max、min等。
\end{itemize}
\par 概括总结一下:在多维分析中,关注的内容通常被称为度量(Measure),而把限制条件称为维度(Dimension)。多维分析就是对同时满足多种限制条件的所有度量值做汇总统计。包含度量值的表被称为事实表(Fact Table),描述维度具体信息的表被称为维表(Dimension Table),同时有一点需要注意:并不是所有的维度都要有维表,对于取值简单的维度,可以直接使用事实表中的一列作为维度展示。
\par 下面是Mondrian中一个简单的Schema文件,其中包含一个名为“Sales”的Cube,立方体中有两个维度:“Gender”和“Time”,两个度量值:“Unit Sales”和“Store Sales”。
\begin{verbatim}
<Schema>
<Cube name="Sales">
<Table name="sales_fact_1997"/>
<Dimension name="Gender" foreignKey="customer_id">
<Hierarchy hasAll="true" allMemberName="All Genders" primaryKey="customer_id">
<Table name="customer"/>
<Level name="Gender" column="gender" uniqueMembers="true"/>
</Hierarchy>
</Dimension>
<Dimension name="Time" foreignKey="time_id">
<Hierarchy hasAll="false" primaryKey="time_id">
<Table name="time_by_day"/>
<Level name="Year" column="the_year" type="Numeric" uniqueMembers="true"/>
<Level name="Quarter" column="quarter" uniqueMembers="false"/>
<Level name="Month" column="month_of_year" type="Numeric" uniqueMembers="false"/>
</Hierarchy>
</Dimension>
<Measure name="Unit Sales" column="unit_sales" aggregator="sum" formatString="#,###"/>
<Measure name="Store Sales" column="store_sales" aggregator="sum" formatString="#,###.##"/>
<Measure name="Store Cost" column="store_cost" aggregator="sum" formatString="#,###.00"/>
<CalculatedMember name="Profit" dimension="Measures" formula="[Measures].[Store Sales] - [Measures].[Store Cost]">
<CalculatedMemberProperty name="FORMAT_STRING" value="$#,##0.00"/>
</CalculatedMember>
</Cube>
</Schema>
\end{verbatim}
\subsection{数据预处理}
\par 不正确,不完整和不一致的数据是商场中数据的普遍特点,如许多数据库表元组在某些属性上根本没有值,有些期望获得的属性没有被记录,对同一款商品,不同商场的命名不一样。当用户不希望暴露个人信息的时候,可能故意输入不正确的生日日期。对一个大型商场而言,某个用户的家庭住址可能早就过期了,但针对不同的数据挖掘任务,其满意度也不同,销售人员不太满意,但市场分析人员对这个家庭住址就不太看重。
\begin{itemize}
\item 数据清理用来清除数据中的噪声,纠正不一致。数据清理通过填写缺失的值,光滑噪声数据,识别或删除离群点,也就是清除脏数据的过程,脏数据可能会使某些挖掘算法陷入混乱,导致不可靠的输出。ETL(Etractrion,Transform,Loading)工具允许用户说明简单的变换,如将字符串“列宁格勒”用“圣彼得堡”代替。
\item 数据集成将数据由多个数据源合并成一个一致的数据存储,如数据仓库。代表同一概念的属性在不同的数据库中可能具有不同的名字,关于顾客标识的属性在一个商场数据库中是\textsl{customer\_id},而在另一个数据库中为\textsl{cust\_id},命名的不一致还可能出现在属性值中,例如,方便面在第一个数据库中登记为\textbf{方便面},在第二个数据库中登记为\textbf{泡面},在第三个数据库中登记为\textbf{速食面},某些属性值可能是由其它属性值导出的,例如选修某课程的人数。
\item 数据归约可以通过聚集,删除冗余特征或者聚类方法来降低数据的规模,得到数据的简化表示,小很多,但能产生同样的分析结果。经常用的方法有:小波变换,主成分分析,属性子集选择(去掉不相关的属性)和属性构造(从原来的属性集中生成更有用的小属性集),使用参数模型(回归和对数线性模型)或非参数模型(直方图,聚类,抽样)等。
\item 数据变换可以用来把数据压缩到较小的空间,如0.0到1.0,可以提高涉及距离度量的挖掘算法的准确率和效率。例如,涉及诸如神经网络,最近邻分类或聚类这样的基于距离的挖掘算法。对不同的属性,例如年龄和年薪,年薪的取值范围远远大于年龄,因此,属性值必须规范化。同样的,数据必须进行概念分层,属性的原始值被区间或更高层的概念取代,年龄的原始值可以用较高层的概念(如少年,青年,壮年,老年)取代。
\end{itemize}
\subsection{数据变换与数据离散化}
\par 数据必须变换成统一格式,使得数据挖掘过程更加有效,数据变换策略包括如下几种:
\begin{enumerate}[(1)]
\item 光滑,去掉数据中的噪音,包括分箱,回归和聚类,
\item 属性构造,可以由给定的属性构造出新的属性,并添加到属性集中,
\item 聚集,对数据进行汇总,例如,可以聚集日销售数据,计算每月和每年的销售量,这一步用来为多个抽象层的数据分析构造数据立方体。
\item 规范化,把属性数据按比例缩放,使其落入一个特定的小区间,如$[-1.0 \sim 1.0]$内
\item 离散化:数值属性的原始值(例如年龄)用区间标签(例如,$[0 \sim 10]$,$[11 \sim 20]$,$\cdots$)或概念标签(\textbf{年轻},\textbf{成熟},\textbf{老者})代替,这些概念标签能进一步组成概念分层,即属性的扩展。
\item 由标称数据产生概念分层:属性,如\textbf{街道},可以泛化到较高的概念层,如\textbf{城市},\textbf{省份},\textbf{国家}等
\end{enumerate}
\subsection{数据仓库与数据挖掘}
\par 为设计有效的数据仓库,必须理解和分析商务需求,并构造一个商务分析框架。建立数据仓库的关键有:如何构造一个提取程序,将数据由商场中的操作数据库转换到数据仓库;如何构造一个仓库刷新软件,合理地保持数据仓库相对于当前商场数据库的实时性,即数据应保持尽可能的同步。
\begin{enumerate}[(1)]
\item 选取多个待建模的主题,以主题为中心,构造数据集市(以星形模式构建主题);
\item 选取处理的粒度,在事实表中是数据的原子级(例如,单个事务,数据库一天的快照);
\item 选取用于每个事实表记录的维,典型的维包括时间,商品,顾客,供应商,仓库,事务类型和状态;
\item 选取将安装在每个事实表记录中的度量,典型的度量(measure)函数有count(),sum(),min(),max()等。
\end{enumerate}
\par 最初,数据仓库主要用于产生报告和预先定义的查询,渐渐地,它用于分析汇总和详细数据,结果以报表和图标形式提供,稍后,数据挖掘用于决策,进行多维分析和复杂的切片与切块操作,最后,使用数据挖掘,数据仓库可用于知识发现。因此,定义在数据仓库上的工具包括:访问与检索工具,数据库报表工具,数据分析工具和数据挖掘工具。一般来说,数据仓库的应用,进化步骤如下:\framebox{信息处理}$\longrightarrow$\framebox{分析处理}$\longrightarrow$\framebox{数据挖掘},其中信息处理支持查询和基本的统计分析,并使用表或图进行报告,分析处理支持数据仓库的多维数据分析,数据挖掘支持知识发现,包括找出隐藏的模式和关联,构造分析模型,进行分类和预测。
\par 数据挖掘比OLAP的功能要广泛许多,OLAP的目标是简化和支持交互数据分析,是用户指导下的汇总和比较,而数据挖掘不仅执行数据汇总和比较,也执行关联,分类,预测,聚类,时间序列分析和其它数据分析任务。数据仓库包含海量数据,OLAP服务器要在数秒内回答决策支持查询,数据仓库系统要支持高效的数据立方体计算技术,存取方法和查询处理技术,

